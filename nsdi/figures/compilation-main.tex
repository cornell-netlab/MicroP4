\documentclass[a4paper,18pt]{article}

\usepackage[english]{babel}
\usepackage[T1]{fontenc}
\usepackage[ansinew]{inputenc}
\usepackage{libertine}
\usepackage[italic]{mathastext}
\usepackage{amsmath,amsthm,amsfonts}
\usepackage{xcolor}
\usepackage{tikz}
\usepackage{subcaption}
\usepackage{inconsolata}

\usetikzlibrary{shapes,arrows,shadows}
\usetikzlibrary{arrows.meta}
\usetikzlibrary{shapes.geometric}
\usetikzlibrary{automata,positioning,arrows,matrix,backgrounds,calc}
\usetikzlibrary{decorations.text}
\usetikzlibrary{decorations.pathmorphing}

\usepackage[active,tightpage]{preview}
\usepackage{xparse}
\PreviewEnvironment{tikzpicture}
\setlength\PreviewBorder{0pt}%

\usetikzlibrary{fit}					% fitting shapes to coordinates
\usetikzlibrary{backgrounds}	% drawing the background after the foreground
\usetikzlibrary{calc}

\renewcommand*{\familydefault}{\sfdefault}% Let's have a sans serif font

\begin{document}

\tikzstyle{userdefined}=[rectangle, draw=black, minimum height=4ex, text width=7em, text centered, inner sep=1pt, anchor=west]
\tikzstyle{intermediate}=[rectangle, draw, dashed, thick, minimum height=4ex, text width=7em, text centered, inner sep=1pt, anchor=west]
\tikzstyle{output}=[rectangle, rounded corners, draw, minimum height=4ex,text width=5em, text centered, inner sep=1pt, anchor=west]
\tikzstyle{provided}=[rectangle, rounded corners, fill=cyan!10, minimum height=4ex,text width=5em, text centered, inner sep=1pt, anchor=west]

\tikzstyle{code}=[userdefined]
\tikzstyle{compiler}=[provided]
\tikzstyle{textbox}=[text width=2em, text centered, text width=3em]
\tikzstyle{line}=[draw, very thick, color=black!75, -latex']
\tikzstyle{empty}=[]

\newcounter{cntr}
\begin{tikzpicture}[>=latex]
% the shapes
  \node (input) at (0,0) [code] {\texttt{main.$\mu$p4}};
  \node (l2api) at (0.1,-0.9) [intermediate, draw=green!50!black] {L2 $\mu$P4 IR};
  \node (l2ir) at (0,-1) [intermediate, draw=red!50!black, fill=white] {L2 $\mu$P4 IR};

  \node (l3api) at (0.1,-1.9) [intermediate, draw=green!50!black] {L3 $\mu$P4 IR};
  \node (l3ir) at (0,-2) [intermediate, draw=red!50!black, fill=white] {L3 $\mu$P4 IR};

  \node (compiler) at ($(input.east)+(2,0)$) [compiler] {$\mu$P4C};
  \node (main-api) at ($(compiler.east)+(1.1,0.1)$) [intermediate, draw=green!50!black] {main.p4};
  \node (mainp4) at ($(compiler.east)+(1,0)$) [intermediate, draw=blue!50!black,fill=white] {main.p4};

  \node (p4c) at ($(mainp4.south)+(0,-0.8)$) [compiler, text width
    =8em, anchor=center] {P4C or target-compiler};
  
  \node (exe) at ($(p4c.south west)+(-2.5,-0.7)$) [output, text width=8em] {target-specific exe};
  \node (api) at ($(p4c.south west)+(1,-0.7)$) [output, text width=8em] {control API};
  \node () at ($(compiler.south east)+(0.3,-0.3)$) [anchor=east] {\texttt{\--\--arch=<real target>}};

\path [line] (input.east) -- (compiler.west);
\path [line] (l2ir.east) -- ($(l2ir.east)+(0.4,0)$) -- ($(input.east)+(0.4,0)$) -- (compiler.west);
\path [line] (l3ir.east) -- ($(l3ir.east)+(0.4,0)$) -- ($(input.east)+(0.4,0)$) -- (compiler.west);
\path [line] (compiler.east) -- (mainp4.west);
\path [line] (mainp4.south) -- (p4c.north);
\path [line] (p4c.south) -- (exe.north);
\path [line] (p4c.south) -- (api.north);
\end{tikzpicture}
\end{document}
