\documentclass[10pt,sigconf,letterpaper,anonymous]{acmart}


%%
%% \BibTeX command to typeset BibTeX logo in the docs
\AtBeginDocument{%
  \providecommand\BibTeX{{%
    \normalfont B\kern-0.5em{\scshape i\kern-0.25em b}\kern-0.8em\TeX}}}

%% Rights management information.  This information is sent to you
%% when you complete the rights form.  These commands have SAMPLE
%% values in them; it is your responsibility as an author to replace
%% the commands and values with those provided to you when you
%% complete the rights form.
\setcopyright{acmcopyright}
\copyrightyear{2019}
\acmYear{2019}
%% \acmDOI{TBD}

%% These commands are for a PROCEEDINGS abstract or paper.
\acmConference{CoNEXT '19}{December 9-12, 2019}{Orlando, Florida, USA}
%% \acmBooktitle{}
%% \acmPrice{15.00}
%% \acmISBN{978-1-4503-9999-9/18/06}


%%
%% Submission ID.
%% Use this when submitting an article to a sponsored event. You'll
%% receive a unique submission ID from the organizers
%% of the event, and this ID should be used as the parameter to this command.
%%\acmSubmissionID{123-A56-BU3}


\begin{document}

%%
%% The "title" command has an optional parameter,
%% allowing the author to define a "short title" to be used in page headers.
\title{To be decided}

%%
%% The "author" command and its associated commands are used to define
%% the authors and their affiliations.
%% Of note is the shared affiliation of the first two authors, and the
%% "authornote" and "authornotemark" commands
%% used to denote shared contribution to the research.


\author{Julius P. Kumquat}
\affiliation{\institution{The Kumquat Consortium}}
\email{jpkumquat@consortium.net}

%%
%% By default, the full list of authors will be used in the page
%% headers. Often, this list is too long, and will overlap
%% other information printed in the page headers. This command allows
%% the author to define a more concise list
%% of authors' names for this purpose.
\renewcommand{\shortauthors}{Trovato and Tobin, et al.}

%%
%% The abstract is a short summary of the work to be presented in the
%% article.
\begin{abstract}
 

 
 
\end{abstract}

%%
%% The code below is generated by the tool at http://dl.acm.org/ccs.cfm.
%% Please copy and paste the code instead of the example below.
%%
\begin{CCSXML}
<ccs2012>
 <concept>
  <concept_id>10010520.10010553.10010562</concept_id>
  <concept_desc>Computer systems organization~Embedded systems</concept_desc>
  <concept_significance>500</concept_significance>
 </concept>
 <concept>
  <concept_id>10010520.10010575.10010755</concept_id>
  <concept_desc>Computer systems organization~Redundancy</concept_desc>
  <concept_significance>300</concept_significance>
 </concept>
 <concept>
  <concept_id>10010520.10010553.10010554</concept_id>
  <concept_desc>Computer systems organization~Robotics</concept_desc>
  <concept_significance>100</concept_significance>
 </concept>
 <concept>
  <concept_id>10003033.10003083.10003095</concept_id>
  <concept_desc>Networks~Network reliability</concept_desc>
  <concept_significance>100</concept_significance>
 </concept>
</ccs2012>
\end{CCSXML}

\ccsdesc[500]{Computer systems organization~Embedded systems}
\ccsdesc[300]{Computer systems organization~Redundancy}
\ccsdesc{Computer systems organization~Robotics}
\ccsdesc[100]{Networks~Network reliability}

%%
%% Keywords. The author(s) should pick words that accurately describe
%% the work being presented. Separate the keywords with commas.
\keywords{datasets, neural networks, gaze detection, text tagging}

%% A "teaser" image appears between the author and affiliation
%% information and the body of the document, and typically spans the
%% page.

%% \begin{teaserfigure}
%%  \includegraphics[width=\textwidth]{sampleteaser}
%%  \caption{Seattle Mariners at Spring Training, 2010.}
%%  \Description{Enjoying the baseball game from the third-base
%%   seats. Ichiro Suzuki preparing to bat.}
%%   \label{fig:teaser}
%% \end{teaserfigure}

%%
%% This command processes the author and affiliation and title
%% information and builds the first part of the formatted document.
\maketitle

\section{Introduction}

Software-Define Networking paradigm provides a great flexibility to control and manage network devices by separating control and data planes.
Using APIs, Control plane can configure objects in data plane at run-time or compile time to achieve desired packet processing behavior or modify it.
Adding into that, P4, a data plane programming language, allows to describe data plane of packet processing logic required to realize a network function and expose APIs to configure the objects the in data plane.
<<1 line on programmable blocks and on re-configurable hardware, Flexpipe, RMT >>

More often, programmers need to describe data and control plane logic of network functions as different programs and in different languages.
This, by design split of logic and execution control flow across multiple heterogeneous programs, increases development, test and deployment complexity of network functions.
Also, it requires novel mechanisms to build complex network functions by reusing independently developed and tested data and control plane code.
However, P4 mandates to write a monolithic data plane program and carefully configure the data plane objects to build a application processing various protocols and performing multiple network functions.
For example, switch.p4~\cite{switch.p4} has control blocks defined to processes different protocol headers and network functions(e.g., l2 switching, l3 routing etc.,). 
But, the control blocks globally share different types of metadata structures and parsed headers. 
Without understanding implementation details of the program, reuse of the code is difficult due to lack of clear interface and abstraction.
Existing data plane programming paradigm needs modularity that allow programmers to expose interface to reuse code written to process packet at any granularity of functionalities while abstracting away the implementation details.

<<Draw a diagram that can be shared with this text and section 2>>
Let's consider a simple scenario. A program, l3.p4, (defines a callable entity that) parses IPv4 header from packets, performs longest-prefix match and determines next hop. 
Moreover, it decrements the ttl field and deparse the packet. 
Another program, l2.p4, processes the same packet and takes the next hop as input argument, parses Ethernet header, matches on next-hop and modifies ethernet addresses.
Finally, it deparses the packet and sends on appropriate port.
In this example, l3.p4 is not generating a functionally correct packet to forward on wire. 
However, it can be reused with different layer-2 forwarding mechanism or even with MPLS and create functionally correct packet to forward on wire.
Such fine-grained packet processing modules enable code reuse and modular control over data plane objects, thereby facilitating incremental development of network functions. 



Previous work, HyPer4~\cite{Hancock:2016:HUP:2999572.2999607}, HyperV~\cite{8038396} use virtualization to support modularity.
P4Visor~\cite{Zheng:2018:PLV:3281411.3281436} supports testing specific composition operators(A-B and Differential) by merging P4 programs using compiler techniques.
In both approaches, minimal composable unit is a data plane program of a network function(e.g., switching, routing etc.,). 
Hence, it does not allow to incrementally develop and enrich a network function by reusing code to support more protocols.
Also, these approaches lacks inter-module communication mechanism(e.g., next hop in above example) except via packets.
Encapsulating customized headers inside the packet may allow such communication, but that would require to know
implementation details of deparser in one module to write complimentary parser in other and vice versa. 


In this paper, we present...
A composable(Modular or better name) architecture(CSA) with simple layout of packet processing blocks to abstract away complex interaction and packet execution flow of the target devices.
The architecture allows to write fine-grained packet processing functions(packages?) and define interface to expose them as callable modules(functions, package..?).
P4 programmers can make reuse of the code by invoking the modules using their interfaces without knowing their implementation details.
\begin{itemize}
 \item Compiler Midend to link all the programmable blocks compose them as dictated by their call location in execution-control of the source program.
 \item and transform complete CSA specific P4 program to any target architecture (e.g., v1model of BMV2 or PSA)
\end{itemize}

%  
% Contribution in this paper,
% 
%  P4 programs adhering CSA are translated to other architectures(e.g., v1model, PSA) pertaining to software or hardweare target. 
%  
%  we developed backend and blah blah  
%  
%  
%  Techniques to transform parser and deparser architecture blocks of P4 programs into control-blocks comprising match-action tables.
% Compiler midend to merge packet processing functionality described in other p4 programs.

<<Paper outline para  >> rest of the paper is arranged....


\section{An example scenario}

Compilation procedure.. with diagram

Take the above example and explain how to use

how does it work?

how to write fine-grained packet processing code, how to define interface etc..

What happens to control plane APIs?

\section{Composable Switch Architecture}

Rationale

P4 programs are tightly coupled with target architecture and layout of programmable blocks.

Lacks unified abstraction for heterogeneous packet processing blocks in hardware  .

That makes it difficult to compose..

Need to minimize hardware details exposed in architecture and provide better abstraction to programmer to write packet processing blocks.

So, we need simple packet processing pipeline

ingress-egress, buffer, packet-replication engine etc can be abstracted away from the programmer


\section{Compiler}

\subsection{Parser/Deparser Transformation}
converting parser and deparser into control blocks

\subsection{Architecture Translation}
Explain constraint.. take an example and show split of control-flow graph



\section{implementation}

\section{Related Work}

\section{Conclusion}

%%
%% The acknowledgments section is defined using the "acks" environment
%% (and NOT an unnumbered section). This ensures the proper
%% identification of the section in the article metadata, and the
%% consistent spelling of the heading.
\begin{acks}
...
\end{acks}

%%
%% The next two lines define the bibliography style to be used, and
%% the bibliography file.
\bibliographystyle{ACM-Reference-Format}
\bibliography{main}

%%
%% If your work has an appendix, this is the place to put it.
%% \appendix



\end{document}
\endinput
%%
%% End of file `sample-sigconf.tex'.
