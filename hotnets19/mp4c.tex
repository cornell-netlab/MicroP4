\section{$\mu$P4C Compiler}
\label{section-mp4c-compiler}


\subsection{Parser Block Transformation}
\label{subsection:parser-block-transformation}
P4 parser blocks describe parse graphs as state machines.
Real target devices contain programmable parser module that can be programmed using parse graphs.
From the design of programmable parser~\cite{6665172}, we make following observations.
Programmable parsers are implemented using buffer, state machine logic, Ternary Content-Addressable Memories (TCAM) and Action RAM. 
TCAM matches values of current state, fields or variables to identify next state. 
Based on match, headers are copied from bit stream and current state is modified.
Programmable parsers essentially perform repeated match and action.
Also, we note that network packets are of finite length, hence they can be parsed in a finite number of ways.
Successful parsing of a packet is essentially finding a match for a finite number of bytes from finite set of values.
However, we need to extract all the bytes that might be required to perform match-actions from packets' bit streams.


We compute number of bytes required to extract in section~\ref{subsubsection:computing-size-of-byte-array}, followed by an approach to convert parser without loops and variable length headers, called \textit{simple parsers}, into a series of match-action tables.
Next, we explain loops unrolling and variable length headers removal techniques transforming every parser into a simple parser.
Finally, we discuss an optimization to reduce number of MATs to one.


\subsubsection{Computing Size of Byte Array}
\label{subsubsection:computing-size-of-byte-array}
Every~\texttt{extract} method call statement in parser states advances bit index by number of bits equal to the size of the header type of the instance passed as the argument.
We perform symbolic execution of parser, control and deparser blocks to determine size of byte array buffer by evaluating calls of ~\texttt{extract} method of core library extern~\texttt{packet\_in}, ~\texttt{setValid} and ~\texttt{setInvalid} of P4 header types and~\texttt{emit} method of another core library extern~\texttt{packet\_out}.
Symbolic execution of a parser block enumerates every possible path from the start to the accept state in the parse graph and computes total number of bytes extracted on every path.
We define length $l_{p}(x)$ of a path $x$ from the start to the accept state in the parse graph of program $p$'s parser as total number of bytes extracted.
And, the input buffer size for program $p$'s parser as $\mathcal{I}(p) = \max_{x}(l_{p}(x))$. 


Programs may increase or decrease size of packets, therefore, considering only parser buffer size is not enough.
We analyse control and deparser blocks to determine maximum increase and decrease in packet size by the program.
Initially, we set the validity bit of all the headers instances that could be extracted by the parser. 
We perform symbolic execution of control blocks to evaluate validity of each header instance on every path of the control flow graph.
Any header instance not emitted in all the paths of deparser block is considered invalid, because such header instances will not increase size of the packet.
We compute maximum and minimum number of bytes that can be emitted by program $p$ and denote them as $\mathcal{O}_{m}(p)$ and $\mathcal{O}_{n}(p)$, respectively.
We define maximum decrease and increase in packet size by program $p$ as $\delta(p)$ and $\Delta(p)$, as shown in (\ref{decrease}) and (\ref{increase}).
\begin{align}
\delta(p)\; =& \; \begin{cases}
\mathcal{I}(p) - \mathcal{O}_{n}(p), & \text{ if } \mathcal{I}(p) > \mathcal{O}_{n}(p), \\
0, & \text{ else }
\end{cases} \label{decrease} \\
\Delta(p) \; =& \; \begin{cases}
\mathcal{O}_{m}(p) - \mathcal{I}(p), & \text{ if } \mathcal{O}_{m}(p) > \mathcal{I}(p), \\
0, & \text{ else } \label{increase}
\end{cases}
\end{align}

To process packets by a sequence of $N$ programs, we define extract length, $\mathcal{E}l_{S}$ and buffer length $\mathcal{B}l_{S}$, as shown in (\ref{extract-length-seq}) and (\ref{buffer-length-seq}).
$\mathcal{E}l_{S}$ denotes the maximum number of bytes that may be extracted as cumulative effect of deparsers and parsers of all the program in a sequence.
$\mathcal{B}l_{S}$ denotes the maximum number of bytes that may be emitted as cumulative effect of the deparsers of all the programs in a sequence.
Similarly, we define extract length, $\mathcal{E}l_{P}$ and buffer length $\mathcal{B}l_{P}$, as shown in (\ref{extract-length-par}) and (\ref{buffer-length-par}), to process packets by $N$ programs in parallel.
\begin{align}
\mathcal{E}l_{S} \; =& \; \max_{i} \left\{ \left( \sum_{j=0}^{j<i} \delta(j) \right)+ \mathcal{I}(i) \right\},&\;\;\;i  \in [0,N] \label{extract-length-seq} \\
\mathcal{B}l_{S} \; =& \; \left( \sum_{i=0}^{N} \Delta(i) \right)+ \mathcal{E}l_{S} & \label{buffer-length-seq} \\
\mathcal{E}l_{P} \; =& \; \max_{i} \left\{ \mathcal{I}(i) \right\},&\;\;\;i  \in [0,N] \label{extract-length-par} \\
\mathcal{B}l_{P} \; =& \; \max_{i} \left\{ \mathcal{I}(i) + \Delta(i) \right\},&\;\;\;i  \in [0,N]  \label{buffer-length-par}
\end{align}


\subsubsection{Simple Parser to MATs}
\label{subsubsection:simple-parser-to-mats}
% We create action stage using parserStatements and match stage using ~\texttt{transitionStatement}.
Every path enumerated by symbolic execution of a parser consists of evaluated instances of the parser states.
A parser state can be a part of multiple paths, thereby having multiple instances.
<<>Diagram>
% Every evaluated instance of a state provides a set of extracted (valid) header instances and their start bit indices corresponding to the state and the path.
P4 parser states consist of~\texttt{parser\-Statements} and~\texttt{transition\-Statement}.
The select expression in transition statement could be a header field, metadata or local variable declared in the parser.
The value of select expression of a state may depend on its ancestors' parser statements. <<as shown in diagram>>
Therefore, we perform Forward Substitution on select expressions in evaluated instances of states
% (https://dl.acm.org/citation.cfm?id=7904)
on each path and eliminate such data dependency.

We synthesise local binary variables, called $visit$, for each parser state to track the state transition of the parser's FSM.
For every evaluated instance of a parser state, we synthesise an action comprising its~\texttt{parser\-Statements} and replace extract method call statements to assignment statements.
The assignment statements copies bytes from the buffer array to header instances' fields according to their sizes.
Next, we add pop method call with the header size as the argument to remove the header from the byte array.
We insert~\texttt{setValid} method call statement for the extracted header instances.
We add an assignment statement to set $visit$ variable associated with the parser state.


For every parser state, we create a match key comprising key-fields from sets of keys. 
$(1)$ $visit$ variable assocociated with the set of all its ancestors and the state itself and
$(2)$ union of select expression of all of its evaluated instances.
In most cases, the union of select expression would have a single key-field, unless forward substitution has produced different expressions in the state's evaluated instances.

The~\texttt{start} being a special state has only one evaluated instance.
We create an action, called~\texttt{start\_action}, from the only instance of the start state.
Next, we visit parser states in topological order. 
We create match key for the current parser state as described above.
~\texttt{keysetExpression} and all possible paths represented $visit$ bit vector match-key entries are synthesised.
For each match key entry, we use appropriate action synthesised by next state's evaluated instance.

The apply body of the control block consists of an action call invalidating all the header instances, followed by the ~\texttt{start\_action} call and apply calls to the MATs created for parser states in topological order.



\subsubsection{Variable Length Headers and Loops Elimination}
\label{variable-length-headers-loops-and-elimination}
The two-argument extract method is replaced with a synthesised sub-parser with three arguments.
First to arguments are the same as two-argument extract method and the third arguments is maximum size of variable field.
The start state of the sub-parser contains a~\texttt{transition\-Statement} with a~\texttt{select\-Expression}.
The sub-parser contains a state for each possible value of variable field size that extracts constant number of bits using single argument extract method and transits to accept state.
The~\texttt{select\-Expression} of the start state uses the variable field size parameter(the second parameter) to match on and transit to the corresponding next state.

How to unroll loops in parser? 
Solution: First, we need to find loops. 
If there is a loop, all( or at least one?) the extract stmts in the loop must(should ?) on header stack.
yet to find a proper general solution.


\subsection{Deparser Block Transformation}
\label{subsection:deparser-block-transformation}
Recall that deparser blocks are specialized control blocks having~\texttt{packet\_out} extern as one of the arguments.
The extern's ~\texttt{emit} method inserts bits on packets' bit-stream, if the header instance provided in the argument is valid, else no operation is performed.
We precisely perform the same operation on byte array buffer, but first we create a match-action table to push fixed number of bytes on the byte array buffer.
Recall that the program's parser transformation pops the total number of extracted bytes from the byte array for each packet.
% If the program's parser extracts less than $\mathcal{E}l$ for a particular packet, there are valid data on byte array.

The key of the table are created using valid bit of each header instance.
The table contains entries enumerating all possible combination of validity of the headers and action corresponding to each match key value to push the number of bytes equals to sum of lengths of all valid headers.


The deparser block may contain P4 language constructs like~\texttt{if-else} and~\texttt{switch} statements in addition of~\texttt{emit} method call statements.
Therefore, deparser block may have multiple execution paths emitting the same header instances. 
We derive a directed-acyclic-graph,~\texttt{emit graph}, with each node representing a emit call from the control flow graph of the deparser control block.
We synthesise a match-action table for every~\texttt{emit} method call in deparser blocks.
Similar to one bit $visit$ variable for parser states, we create one bit variable, $emit$, for each node in the emit graph.
The match key for the match-action table for an emit node comprises valid bit of the header instances passed as argument to its ancestors.
In addition, the match key includes $emit$ variables associated with the ancestors.
The idea behind introducing $emit$ variable is to record execution of previous emit calls in the control path and copy header instance  of emit call at appropriate location in the byte array buffer.



\subsection{(De)parser Control Block Optimization}
The transformation of parser and deparser blocks into match-action control blocks induces huge cost in terms of number of binary variables in data plane and multiple match-action tables.
Also, number of entries in the tables are of exponential order of binary variable fields in the tables.
In this section, we describe optimization methods to reduce per state match-action tables in transformed parser blocks into a single match-action table and eliminate all the binary variables.
 
 


\subsection{Cross-Architecture Code Translation}

